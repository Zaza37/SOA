\documentclass[a4paper]{article}

\usepackage{hyperref}
%\hypersetup{
%colorlinks=false,              % bool: Liens colorés
%pdfborder={0 0 0}             % Ne pas encadrer les liens
%}
\usepackage[utf8]{inputenc}  
\usepackage[francais]{babel}  
\usepackage[top=2cm, bottom=2cm, left=2cm, right=2cm]{geometry}
\usepackage{graphicx}
\usepackage[final]{pdfpages} 
\usepackage{rotating}
\usepackage{eurosym}
\usepackage{lscape}
\usepackage{float}
% définir les commandes ici


% s'il y a beaucoup de commandes et de packages à inclure n'h&ésitez pas
% à mettre tout ça dans un fichier include.tex et l'inclure
% \input{include.tex}


\begin{document}

%------------------------------------- Page de titre
\begin{titlepage}
~ 
\vfill
	\begin{center}
		\begin{Huge}
		SOA : Document bilan\\
		\end{Huge} 
\vfill
		\textbf{Hexanome 4211 :} 
		\\Sandra \bsc{Mondain}, Elisa \bsc{Abidh}, 
		\\Gaël \bsc{Motte}, Armand \bsc{Rossius}, 
		\\Rémi \bsc{Fradet}, Nicolas \bsc{Silva}, Julien \bsc{Levesy}\\

\vfill		
		\begin{Large}
		Avril 2011
		\end{Large}
\vfill
	\begin{tabular}{|c|c|c|c|c|}
 	 \hline
 	 %Auteur du Document & Responsable Validation & Phase & Etat & Avancement \\
 	 \hline
 	 
 	\hline
 	
	\end{tabular}
\vfill	
	\end{center}
\vfill
\end{titlepage}
%----------------------------------------------------

%--------------------------------- Table des matières
\newpage
\tableofcontents
\newpage
%----------------------------------------------- Plan

\input{Init.tex}



\end{document}

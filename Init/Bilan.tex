\section {Charge prévue-réalisée}

\begin{figure}[H]
	\begin{center}
	% l b r t
		\includegraphics[scale=0.8,clip, trim = 20mm 10mm 65mm 5mm]{SuiviTache2.pdf}
		\caption{Planning : charge prévue - charge réelle}
	\end{center}
\end{figure}

\begin{figure}[H]
	\begin{center}
	% l b r t
		\includegraphics[scale=0.8,clip, trim = 0mm 0mm 0mm 0mm]{bilanSOA.jpg}
		\caption{charge prévue - charge réelle}
	\end{center}
\end{figure}

On remarque une différence importante entre le planning prévisionnel et le planning réel d'une part en nombre d'heures et d'autre part en nombre de personnes affectées par tâche. Les explications sont multiples. Dans un premier temps en ce qui concerne le nombre de personne par tâche, les membres du groupe ne mettent pas le même nombre d'heures pour faire les tâches, il a donc fallu répartir les tâches en fonction de l'avancement de chacun. D'autre part certaines personnes ont des préférences avec certaines tâches du projet, pour satisfaire tout le monde et motiver les membres du groupe à travailler il a fallu tenir compte des préférences de chacun. Dans un deuxième temps, le nombre d'heures estimées est différent. En effet, on note une grosse différence au niveau des diagrammes de séquence, ce décalage s'explique par le fait d'une part que personne n'avait une bonne connaissance de comment faire ses diagrammes et avait en tête les diagrammes de séquence du projet USDP. D'autre part l'autre problème réside dans les validations : le temps d'attente pour faire valider est très longue et le corps enseignant a une idée très précise de ce qu'ils attendent, on ne peut pas adapter ce qui a déjà été fait avec ce qu'on attend de nous. \\
En revanche certaines tâches ont été faites plus vite que prévu.

\section{Conclusion}
Ce projet était très intéressant car il permettait de découvrir une partie du fonctionnement de la banque.Il était bien encadré m\^eme si chaque groupe devait patienter en moyenne une heure pour pouvoir faire valider ses diagrammes et autres. Ce projet était également bien calibré en effet si le groupe avait été vraiment efficace en séance il n'y aurait pas eu beaucoup de travail hors séance. Par contre l'enchaînement des t\^aches est assez compliqué à effectuer car avec le retard que nous avons eu sur les diagrammes de séquences si nous avions respecté l'enchaînement des t\^aches nous n'aurions pas pu finir les IHM à temps.  
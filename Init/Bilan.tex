\section {Charge prévue-réalisée}
\input{SuiviTache2.pdf}

On remarque une différence importante entre le planning prévisionnel et le planning réelle d'une part en nombre d'heure et d'autre part en nombre de personne affecté par tâche. Les explications sont multiples. Dans un premier temps en ce qui concerne le nombre de personne par tâche, les membres du groupe ne mettent pas le même nombre d'heure pour faire les tâches, il a donc fallu répartir les tâches en fonction de l'avancement de chacun. D'autre part certaine personne ont des préférences avec certaine tâche du projet, pour satisfaire tout le monde et motiver les membres du groupe à travailler il a fallu tenir compte des préférences de chacun. Dans un deuxième temps, le nombre d'heure estimé est différent en effet on note une grosse différence au niveau des diagrammes de séquence, ce décalage s'explique par le fait d'une part que personne n'avait une bonne connaissance de comment faire ses diagrammes et avait en tête les diagrammes de séquence du projet USDP. D'autre part l'autre problème réside dans les validations : le temps d'attente pour faire valider et très longue et le corps enseignant à une idée très précise de ce qu'ils attendent, on ne peut pas adapter ce qui a déjà fait avec ce qu'on attend de nous. \\
En revanche certaine tâche on était faite plus vite prévue.
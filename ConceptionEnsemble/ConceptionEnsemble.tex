\documentclass[a4paper]{article}

\usepackage{hyperref}
%\hypersetup{
%colorlinks=false,              % bool: Liens colorés
%pdfborder={0 0 0}             % Ne pas encadrer les liens
%}
\usepackage[utf8]{inputenc}  
\usepackage[francais]{babel}  
\usepackage[top=2cm, bottom=2cm, left=2cm, right=2cm]{geometry}
\usepackage{graphicx}
\usepackage[final]{pdfpages} 
\usepackage{rotating}
\usepackage{eurosym}
\usepackage{lscape}
\usepackage{float}
% définir les commandes ici

% s'il y a beaucoup de commandes et de packages à inclure n'h&ésitez pas
% à mettre tout ça dans un fichier include.tex et l'inclure
% \input{include.tex}


\begin{document}

%------------------------------------- Page de titre
\begin{titlepage}
~ 
\vfill
	\begin{center}
		\begin{Huge}
		SOA : Dossier de conception d'ensemble\\
		\end{Huge} 
\vfill
		\textbf{Hexanome 4211 :} 
		\\Sandra \bsc{Mondain}, Elisa \bsc{Abidh}, 
		\\Gaël \bsc{Motte}, Armand \bsc{Rossius}, 
		\\Rémi \bsc{Fradet}, Nicolas \bsc{Silva}, Julien \bsc{Levesy}\\

\vfill		
		\begin{Large}
		Avril 2011
		\end{Large}
\vfill

	\end{center}
\vfill
\end{titlepage}
%----------------------------------------------------

%--------------------------------- Table des matières
\newpage
\tableofcontents
\newpage
%----------------------------------------------- Plan

\section*{Intro}

Première étape de la démarche d'urbanisation du système d'information. Nous allons, dans ce livrable, aborder l'étape "Top-Down", qui a pour objectif final l'obtention d'un découpage en bloc services spécifiés et détaillées à partir de l'étude des activités de l'entreprise.\\
Ainsi dans un premier temps, nous allons détailler les activités menées au sein de l'entreprise à partir du découpage en cas d'utilisation fourni, puis nous poursuivrons notre étude par une présentation du découpage en bloc du MCD. Enfin une étude des interactions entre ces blocs sera effectuée par la construction des diagrammes de séquence et du diagramme de collaboration.\\
Ainsi, nous serons en mesure d'évaluer le couplage et le dimensionnement des flux entre ces différents blocs.

\section{Diagrammes d'Activité}
Todo : importer ici les pnggénérés par cacoo
\section{Découpage en Blocs}
\begin{figure}[H]
	\begin{center}
	% l b r t
		\includegraphics[scale=0.4,clip, trim = 5mm 65mm 35mm 35mm]{Includes/SOA-Blocs-1.pdf}
		\caption{Découpage en bloc du MCD Clients et Produits}
	\end{center}
\end{figure}

\begin{figure}[H]
	\begin{center}
		\includegraphics[scale=0.4,clip, trim = 5mm 30mm 3mm 30mm]{Includes/SOA-Blocs-2.pdf}
		\caption{Découpage en bloc du MCD Commercial}
	\end{center}
\end{figure}
qesg
\section{Diagrammes de Séquences}
prout
\section{Diagramme de Collaboration}
pouet


\end{document}
